\documentclass[pdftex,final,a4paper,11pt,notitlepage,halfparskip]{scrreprt}

\usepackage[german,ngerman]{babel}
\usepackage[utf8]{inputenc}
\usepackage[T1]{fontenc}
\usepackage[babel,german=quotes]{csquotes}
%\usepackage{fancybox}
%\usepackage{color}
%\usepackage{xcolor}
\usepackage{hyperref}
%\usepackage{floatflt}
%\usepackage{graphicx}
%\usepackage{amsmath}
%\usepackage{amssymb}
%\usepackage{amsfonts}
%\usepackage{listings}

\setkomafont{caption}{\footnotesize\linespread{1}\selectfont}
\setlength{\abovecaptionskip}{-0.1cm}
\addto\captionsngerman{\renewcommand\figurename{Abb.}}

\title{Beleg\\
Rechnernetze/\\
Kommunikationssysteme}
\author{Jan Losinski}

\begin{document}

\maketitle

\tableofcontents

\chapter{Aufgabe}

\section{Wortlaut}
Schreiben Sie einen Daemon, welcher E-Mails per ESMTP annehmen und weiterleiten 
oder speichern sowie dem jeweiligen Nutzer per POP3 zur Verfügung stellen kann. 
Die Implementierung von ESMTP muss eine Authentisierung des Nutzers vor dem 
Absenden erzwingen. Die weitere Implementierung muss E-Mails annehmen und an die 
passende E-Mail-Domain weiterleiten können, falls es sich bei dem Empfänger 
nicht um einen lokalen Nutzer handelt. Das aussenden und Abrufen von E-Mails 
muss mit einem E-Mail-Programm, wie beispielsweise Evolution oder Thunderbird 
möglich sein.

Randbedingungen:
\begin{itemize}
  \item Implementierung in C, nicht C\# oder C++ oder \dots
  \item weitgehende Modularisierung, beispielsweise in einen POP3-Parser etc.
  \item der Daemon läuft als ein Prozess ohne Threads (oder Forks)
  \item das Mailboxformat darf beliebig sein
  \item das Mailboxformat darf beliebig sein
  \item die Anzahl der Nutzer darf auf 5 beschränkt sein
  \item der Daemon muss ohne weiteres auf den Rechnern im Labor S311 übersetzt 
        werden können und dort laufen
  \item Gruppenarbeiten sind nicht erlaubt
  \item Die Dokumentation des Daemons wie auch der Quellen geht in die Bewertung 
        ein
  \item es dürfen Bibliotheken verwendet werden, beispielsweise SQLITE oder 
        GDBM, soweit diese keine grundlegenden Funktionen von POP3 oder (E)SMTP 
	zur Verfügung stellen.
  \item Die Abgabe des Belegs erfolgt über das SVN. Geprüft wird die letzte 
        Revision. Das SVN ist ab 29.01.2009 0:00 nur noch lesbar. Nachreichungen 
	werden nicht akzeptiert.
\end{itemize}

Programmieren sie nicht einfach darauf los. Beginnen Sie mit einer Konzeption 
der benötigten Programmmodule. Diese Konzeption darf (und sollte auch) in 
Gruppenarbeit erstellt werden.

Der Daemon soll mindestend folgende Kommandozeilenparameter unterstützen
\begin{description}
  \item[-h] Dokumentation der Kommandozeilenparameter und exit
  \item[-V] Informationen zur Version, SVN-Revision und Autor (Name, Vorname, Login)
  \item[-p <Portnummer,Portnummer,Portnummer>] Portnummern mit Komma getrennt
            für (E)SMTP, POP3 und POP3S\\ 
            Voreingestellte Portnummern 25,110,995
  \item[-u <Dateiname>] Datei mit den Nutzernamen als CSV-Datei mit "`TAB"' als 
            Feld-Trenner\\ 
	    Format: \texttt{Login\textbackslash tPasswort\textbackslash
	    tsonstiges\dots\textbackslash n}
\end{description} 

\section{Erfragte Zusätze}
Folgende zusätzliche Fahten zur Aufgabenstellung wurden erfragt:
\begin{itemize}
  \item SMTPS soll nicht implementiert werden.
  \item Die Ein/Ausgabe kann blockierend erfolgen.
  \item Der Mailheader muss nicht geparst werden, auch eine Extraktion von
    Mailaddressen muss nicht erfolgen.
  \item Mailrelay soll nur bei authentifiziertem ESMTP möglich sein.
  \item Die Zustellung lokaler Mails ist ohne authehtifizierung möglich. 
\end{itemize}<++>

\chapter{Umsetzung}
\chapter{Benutzung}

\end{document}
